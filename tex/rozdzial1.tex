\chapter{Wstęp}
\thispagestyle{chapterBeginStyle}
\label{chapter1}


Sieci neuronowe pozwalające na analizę emocji tłumów to wciąż jeszcze słabo przebadany temat. W ciągu ostatnich kilku lat zyskuje jednak zainteresowanie naukowe \cite{Group1}\cite{Group2}\cite{Group3}\cite{Group4}, głównie ze względu na ogromną ilość danych dostępnych na serwisach społecznościowych, zawierających zdjęcia grup osób uczestniczących w różnych wydarzeniach towarzyskich.

Poniższa praca porusza temat uczenia maszynowego, w szczególności aspekty związane z sieciami neuronowymi i ich wykorzystaniem. Praca skupia się na projektowaniu, budowaniu i trenowaniu głębokich konwolucyjnych sieci neuronowych, a także na wykorzystaniu ich w analizie dominujących emocji panujących na zdjęciach tłumów i dużych grup ludzi. Ponadto, w pracy poruszany jest również temat baz danych przechowujących wyżej wspomniane zdjęcia wraz z odpowiadającymi im oznaczeniami wskazującymi dominującą na obrazie emocję, jak również temat aplikacji mającej ułatwić i zautomatyzować proces trenowania sieci neuronowych.

Celem pracy jest zaprojektowanie, zbudowanie i wyszkolenie sztucznej sieci neuronowej zdolnej do analizy emocji tłumów na podstawie zdjęć. Sieć ma być w stanie określić emocję panującą na fotografii, przydzielając ją do jednej z trzech kategorii: emocji pozytywnych, neutralnych bądź negatywnych.

Praca składa się z siedmiu rozdziałów.
Niniejszy wstęp stanowi rozdział \ref{chapter1}.
W rozdziale \ref{chapter2}. zaprezentowano najważniejsze pojęcia z dziedziny uczenia maszynowego, których znajomość jest niezbędna podczas lektury niniejszej pracy. Ponadto, znajduje się tam przegląd literatury związanej z sieciami neuronowymi.
W rozdziale \ref{chapter3}. szczegółowo przeanalizowano rozpatrywany problem, dokładnie sprecyzowano cele pracy, przedstawiono istniejące rozwiązania i opisano zastosowane podejścia do rozwiązania problemu.
W rozdziale \ref{chapter4}. zaprezentowano informacje dotyczące bazy danych, jak również wyjaśniono, dlaczego właśnie ta baza została wybrana. Ponadto, omówiono architekturę konwolucyjnych sieci neuronowych, podstawy jej działa, oraz problemy, z jakimi należy się mierzyć, ze szczególnym naciskiem na problem nadmiernego dopasowania do danych treningowych. Z tego powodu omówiono również technikę augmentacji danych. Przeanalizowano też modele uprzednio trenowane, których wykorzystanie w procesie transfer learningu może wpłynąć pozytywnie na dokładność walidacji. Na koniec opisano architekturę aplikacji mającej na celu usprawnić proces trenowania sieci neuronowych oraz skryptu pozwalającego na wygodne wykorzystywanie wytrenowanej sieci w celu przewidywania emocji. 
W rozdziale \ref{chapter5}. opisano technologie implementacji projektu: wybrany język programowania oraz frameworki. Przedstawiono architekturę stworzonej aplikacji i skryptu. Zaprezentowano implementacje modelu sieci neuronowej, sposób augmentacji danych i rozszerzenia modelu uprzednio trenowanego, wraz z przedstawieniem wyników analiz obrazów przeprowadzanych przez zaimplementowane modele.
W rozdziale \ref{chapter6}. przedstawiono sposób instalacji i wdrożenia aplikacji w środowisku docelowym.
Rozdział \ref{chapter7}. stanowi podsumowanie stanu zakończonych prac projektowych i implementacyjnych  wraz z potencjalnymi, dalszymi kierunkami ich rozwoju.
